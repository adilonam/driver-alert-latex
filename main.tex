\documentclass{report}
\usepackage{hyperref}
\usepackage{listings}
\usepackage{xcolor}

% Define colors for syntax highlighting
\lstset{
  basicstyle=\ttfamily,
  keywordstyle=\color{blue},
  stringstyle=\color{red},
  commentstyle=\color{green},
  showstringspaces=false,
  columns=fullflexible,
  keepspaces=true,
  breaklines=true
}

\title{App to Detect Siren Installation Guide}
\author{Ahmed}
\date{\today}

\begin{document}

\maketitle

\tableofcontents

\chapter{Introduction}
This report provides a step-by-step guide to setting up an application to detect sirens. The guide is divided into two parts: installation of necessary software and creating a standalone app.

\chapter{Part One: Installation Steps}

\section{Step 1: Download Miniforge}
To begin, download Miniforge using the `wget` command.

\begin{lstlisting}[language=bash]
wget https://github.com/conda-forge/miniforge/releases/latest/download/Miniforge3-Linux-aarch64.sh
\end{lstlisting}

\section{Step 2: Install Miniforge}
Run the downloaded script to install Miniforge.

\begin{lstlisting}[language=bash]
bash Miniforge3-Linux-aarch64.sh
\end{lstlisting}

\section{Step 3: Initialize Environment}
Source your bash profile to initialize the environment.

\begin{lstlisting}[language=bash]
source ~/.bashrc
\end{lstlisting}

\section{Step 4: Verify Installations}
Check the installed versions of Conda and Python.

\begin{lstlisting}[language=bash]
conda --version
python --version
\end{lstlisting}

\section{Step 5: Create Conda Environment}
Create a new Conda environment with Python 3.8.

\begin{lstlisting}[language=bash]
conda create -n eva python=3.8
\end{lstlisting}

\section{Step 6: Activate Conda Environment}
Activate the newly created environment.

\begin{lstlisting}[language=bash]
conda activate eva
\end{lstlisting}

\section{Step 7: Clone Git Repository}
Navigate to your home directory and clone the project repository.

\begin{lstlisting}[language=bash]
cd ~
git clone https://github.com/adilonam/Driver-Alert.git
\end{lstlisting}

\section{Step 8: Update System Packages}
Update your system's package list to ensure all packages are up to date.

\begin{lstlisting}[language=bash]
sudo apt update
\end{lstlisting}

\section{Step 9: Install Dependencies}
Install additional required libraries.

\begin{lstlisting}[language=bash]
sudo apt-get install libhdf5-dev
\end{lstlisting}

\section{Step 10: Install Python Requirements}
Navigate to the project directory and install the required Python packages.

\begin{lstlisting}[language=bash]
cd ~/Driver-Alert
python -m pip install -r requirements.txt
\end{lstlisting}

\section{Step 11: Create local variable environment}

\begin{lstlisting}[language=bash]
cd ~/Driver-Alert
cp .env.example .env
\end{lstlisting}

\section{Step 12: Run the Application}
Finally, run the application using Uvicorn.

\begin{lstlisting}[language=bash]
python -m uvicorn main:app --host 0.0.0.0
\end{lstlisting}

\chapter{Part Two: Creating a Standalone App}

\section{Step 1: Make the Script Executable}
Navigate to the Driver-Alert repository an
\begin{lstlisting}[language=bash]
cd ~/Driver-Alert
chmod +x create_service.sh
\end{lstlisting}

\section{Step 2: Execute the Script}
Run the script to create the standalone app. The script will prompt you for two paths.

\begin{lstlisting}[language=bash]
./create_service.sh
\end{lstlisting}

When prompted by the script, provide the following paths:

\begin{lstlisting}[language=bash]
Enter the path to the Python executable: /path/to/python
Enter the path to the Driver-Alert repository: /path/to/Driver-Alert
\end{lstlisting}

Make sure to replace `/path/to/python` and `/path/to/Driver-Alert` with the actual paths on your system.

\section{Step 3: Reboot and enjoy}

\chapter{Conclusion}
This guide has covered the necessary steps to download, install, and set up the application to detect sirens, as well as creating a standalone app. If you encounter any
issues, refer to the project's documentation for further assistance.


\chapter{References}
\begin{itemize}
\item \href{https://github.com/conda-forge/miniforge}{Miniforge GitHub Repository}
\item \href{https://github.com/adilonam/Driver-Alert}{Driver-Alert GitHub Repository}
\end{itemize}

\end{document}
